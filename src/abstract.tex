\begin{abstract}
The exploration of data sets is a frequently discussed issue
in computing and aims at a better understanding of simulated
or measured phenomena. The steps of collecting and storing
the data, which seek to record as much detail about the
observed phenomenon, precede such activity. The exploration
task is challenging due to many aspects. One of them is the
difficulty in identifying which collected data are actually
relevant to the analysis. Another one is related to the lack
of guarantees that the key factors for understanding the
problem have been collected. The interactive transformation
of data is a visual based approach that seeks to solve or
mitigate these problems. However, the available methods in
the literature have limitations in several aspects, such as
complex user interfaces and inflexible interactive
mechanisms. So, the goal of this master's project is to
develop novel visual techniques for the transformation of
data sets. The proposed methodology is based on the use of
biplots and interaction mechanisms to overcome the
limitations of state of the art techniques. We believe that
by using the proposed approach the users will be able to
make the data more representative. Therefore, exploratory
activities may be performed more efficiently and return
better results.
\end{abstract}
