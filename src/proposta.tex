\chapter{Proposta de Trabalho}\label{chap:proposta}

% Os pontos mais importantes são o mecanismo de construção
% (inédito na literatura) e a simplicidade da interface.

\section{Motivação e Objetivo}

No Capítulo~\ref{chap:revisao} foram apresentados os métodos
que buscam modificar os conjuntos de dados para torná-los
mais representativos para o problema em estudo.  Discutiu-se
que os métodos automáticos impedem que os usuários orientem
essas modificações e ao mesmo tempo imponham seus
conhecimentos sobre os resultados.  Apresentou-se as
ferramentas visuais que surgem como uma interessante
alternativa aos métodos automáticos, pois permitem a
interação dos usuários, mas que ainda apresentam certas
limitações em relação às interfaces utilizadas e aos
mecanismos de interação propostos.

O uso de ferramentas visuais como alternativa aos métodos
automáticos não é exclusivo aos trabalhos relacionados
ao aqui proposto. Na verdade, toda a área de
Mineração Visual de Dados~\cite{Wong1999} (MVD),
\emph{Visual Data Mining}, tem como objetivo justamente
envolver os usuários em tarefas que até em tão eram
executadas de maneira totalmente automática. A principal
motivação desta área parte do princípio de quando o usuário
consegue compreender o resultado apresentado por uma
representação visual, então ele confia neste resultado e
consegue tirar melhor proveito das análises.

Uma característica fundamental para ferramentas MVD é manter
a simplicidade em todos aspectos do sistema~\cite{Wong1999}.
No entanto, foi visto no Capítulo~\ref{chap:revisao} que
muitas das ferramentas propostas se baseiam em interfaces
demasiadamente complexas, as quais exigem do usuário um
certo período de treinamento para um uso efetivo. Tendo em
vista que o objetivo das ferramentas visuais é tornar as
análises mais intuitivas, qualquer tipo de obstáculo, como a
necessidade de treinamento do usuário, pode ser desfavorável
ao se comparar com os métodos automáticos.

Um outro aspecto que deve ser levado em consideração para o
desenvolvimento dessas ferramentas é permitir seu uso em
diversos domínios~\cite{Wong1999}. Para isso, diferentes
mecanismos de interação devem ser oferecidos, já que nenhum
mecanismo será capaz de operar otimamente para todas as
aplicações. Porém, no capítulo anterior foi discutido
que nenhuma das ferramentas apresentadas consegue unir em um
único ambiente os principais mecanismos necessários para a
modificação efetiva de conjuntos de dados.

Uma questão que deve ser considerada em ferramentas de
exploração de dados, sejam elas visuais ou não, trata-se de
possibilitar investigações em subconjuntos dos dados. Isto é
importante pois dificilmente o conjunto de dados apresentará
um comportamento global, sendo mais provável que existam
subconjuntos com diferentes características que devem ser
avaliadas localmente~\cite{May2011}.

Levando em consideração os aspectos mencionados nos
parágrafos acima: simplicidade da ferramenta, diversidade de
mecanismos de interação e avaliação global e local dos
dados, este projeto de mestrado se baseia no uso de
\emph{Biplots} para desenvolver uma ferramenta que apresente
essas características e ao mesmo tempo seja capaz de cobrir
as limitações dos trabalhos do atual estado da arte.

Apesar de biplot ser velho...
Acredita-se que o emprego de \emph{Biplots} proporcionará 

Mais especificamente o objetivo deste trabalho pode ser
declarado da seguinte maneira:

\begin{quote}
    \emph{``Este projeto de mestrado tem como objetivo
        desenvolver uma ferramenta visual interativa que
        permita aos usuários modificarem conjuntos de dados
        para torná-los mais representativos. As modificações
        serão realizadas a partir de três principais
        mecanismos de interação: seleção, construção e
        transformação de atributos. Com dados melhor
        representados, os métodos que operam sobre eles,
        como classificadores e agrupadores de dados, devem
        apresentar melhores resultados. Será por meio de uma
        quantificação dessa melhoria que será feita a
    validação da nova ferramenta desenvolvida.''}
\end{quote}

A seguir apresenta-se a metodologia proposta para a criação
das visualizações e para o desenvolvimento de cada um dos
mecanismos interativos.

\section{Metodologia}

% A metodologia proposta por este trabalho é ilustrada pela
% Figura~\ref{fig:met}.

\subsection{Criação da Visualização}

% Um ponto fundamental para o mapeamento dos elementos no
% plano é compreender o erro embutido no processo. O resultado
% ótimo de qualquer método de redução é levar os dados de m em
% p, onde p equivale à dimensionalidade intrínseca dos dados.
% No entanto,  não há garantias de que p equivale à um espaço
% bidimensional, assim ao mapear os dados em um plano, a
% dimensionalidade dos dados será reduzida além da
% dimensionalidade intrínseca e consequentemente haverá perda
% de informação.  Conclui-se que independentemente do método
% de redução adotado, a não ser que a dimensionalidade
% intrínseca dos dados seja equivalente a dois, haverá um erro
% ao mapear os dados em um plano. Dois tipos de erros podem
% ocorrer: pontos similares posicionados distantes entre si
% (falsos negativos), ou pontos diferentes posicionados
% próximos entre si (falsos positivos).  Em uma analogia à
% tarefa de recuperação de informação \emph{information
% retrieval}, esses erros podem ser associados respectivamente
% às medidas de precisão \emph{precision} e revocação
% \emph{recall} (citar 8 e 9 de Kasai).
% Tenho que falar do erro embutido 
% Uma vez definido o cálculo de similaridade entre as
% dimensões, cria-se uma matriz de distâncias 

\subsection{Mecanismos de Interação}

\subsubsection{Seleção}

% Itens e dimensões.
% O propósito principal da visualização da projeção dos
% itens é possibilitar que o usuário crie subconjuntos dos
% dados, por exemplo, possibilitar a remoção de
% \emph{outliers} ou a inspeção de um grupo de interesse.
% Além disso esta visualização serve como um ferramenta
% adicional aos resultados dos classificadores para avaliar
% a separação entre os itens.

% Em tarefas de seleção de característica basicamente
% deseja-se solucionar basicamente dois problemas.  O
% primeiro, chamado de mínimo ótimo (\emph{minimal
% optimal})~\cite{Kohavi1997}, consiste em construir um
% subconjunto dos atributos de entrada evitando ao máximo a
% redundância entre eles.  Um caso prático deste problema
% trata-se da construção de um classificador, onde ao se
% evitar redundância entre os atributos de entrada pode fazer
% com que o método obtenha ganhos tanto no tempo de execução
% quanto na qualidade dos resultados obtidos.  Já o segundo,
% conhecido como todos relevantes (\emph{all
% relevant})~\cite{Nilsson2007}, equivale a encontrar todos os
% atributos que são de algum modo relevantes para a
% compreensão do fenómeno observado.  Uma aplicação real deste
% segundo problema pode ser encontrada no contexto de análise
% de expressões gênicas, onde pode-se, por exemplo,
% identificar quais genes apresentam maior relação com o
% diagnóstico de alguma doença. 
% O mecanismo de seleção aqui proposto viabiliza trabalhar
% sobre esses dois problemas.

\subsubsection{Construção}

% O resultado obtido pela combinação deve ser melhor do que a
% de seleção, porém é também menos intuitivo ao usuário. Mas
% de qualquer modo a abordagem é melhor aplicar PAC em todo o
% conjunto de dados (tirei isso de CHOR, se precisar citar:
% pelo o que pode ser observado no levantamento bibliográfico
% os métodos de combinação tendem a oferecer melhores
% resultados, no entanto...).  

\subsubsection{Transformação}

\subsection{Forma de Avaliação}
 
% Tenho que falar aqui que vou aplicar diferentes métodos de
% redução (eventualmente ter que explicar os que eu escolhi
% cada um, de preferência falar que cada um tem uma
% característica específica) e em seguida vou verificar a
% qualidade da classificação e o tempo de execução do
% classificador.  Vou comparar com outros métodos de redução.
% Em alguns trabalhos~\cite{Joshi2007} a comparação é feita
% sobre a taxa de acertos da classificação com os dados
% originais e com os dados reduzidos com a técnica proposta.
% No entanto, esta validação pode ser questionada, pois
% desconsidera a influencia que outras técnicas de redução
% poderiam ter nos resultados.  Para se comprar com seleção de
% características, \cite{Guyon2003} diz para usar um
% classificador linear (linear SBM) e selecionar as variáveis
% de duas maneiras: por meio de um ranking a partir do
% coeficiente de correlação ou informação mútua; ou pelo uso
% de um método de forward (ou backward) selection.
% \cite{Medeiros2011} apresenta três índices para se comparar
% diferentes técnicas de redução. Aponta que devido a
% heterogeneidade dos métodos, certas convenções devem ser
% estabelecidas para se obter uma comparação justa.  A forma
% mais adotada na literatura para a avaliação de métodos de
% redução de dimensionalidade é a comparação dos erros obtidos
% em tarefas de classificação ao utilizar diferentes técnicas.
% Ao reduzir o número de atributos irrelevantes ou
% redundantes, pode-se melhorar o desempenho computacional e a
% precisão das técnicas operando sobre os dados, como
% agrupadores e classificadores de dados. Pretende-se avaliar
% as contribuições deste trabalho justamente pela
% quantificação do desempenho de tais métodos ao utilizar as
% técnicas desenvolvidas, seguida de uma comparação com
% técnicas já estabelecidas na literatura.  O popular método K
% Means será utilizado neste trabalho para se avaliar Para se
% avaliar o resultado obtido por agrupadores de dados,
% frequentemente utiliza-se a medida da silhueta e o índice I.
% A silhueta mede a .... O índice I é utilizado para...  Falo
% de Sam Precisão, Recall (FAN?) e sei la o que...

\section{Resultados Esperados}

Espera-se que ao término deste trabalho de mestrado,
tenha-se desenvolvido:

\begin{itemize}
    \item Uma ferramenta para
        transformação interativa do espaço de atributos que
        permita seleção, construção e transformação de
        atributos.
\end{itemize}

\section{Resultados Preliminares}

% TODO

\section{Plano de Atividades e Cronograma Previsto}\label{sec:cronograma}

As principais atividades deste trabalho de mestrado são as seguintes:

\begin{enumerate}

    \item Cumprimento dos créditos das disciplinas exigidos pelo programa;

    \item Exame de proficiência em língua inglesa;

    \item Levantamento bibliográfico sobre técnicas de visualização computacional e redução de dimensionalidade;   

    \item Levantamento bibliográfico sobre técnicas visuais interativas para redução de dimensionalidade;  

    \item Adoção e implementação de uma metodologia para o cálculo da similaridade entre dimensões;

    \item Escrita da monografia de qualificação e sua apresentação para uma banca avaliadora;

    \item Implementação de um modelo visual que transmita simultaneamente informações sobre os itens e dimensões de uma base de dados;

    \item Desenvolvimento de mecanismos de seleção e combinação para redução interativa de dimensionalidade;

    \item Desenvolvimento de um mecanismo interativo para transformação do espaço de atributos;

    \item Avaliação dos Resultados;

    \item Redação de artigos científicos e participação em congressos e eventos; 

    \item Escrita da dissertação de mestrado bem como sua apresentação para uma banca avaliadora;

\end{enumerate}

O cronograma de execução das atividades é apresentado na Tabela~\ref{t:atividades}, assumindo um projeto de duração de vinte e quatro meses.

\newcommand{\y}{\color{black}\rule{20pt}{7pt}}
\newcommand{\x}{\hspace*{20pt}}
\renewcommand{\r}{\color{cinza}\rule{20pt}{7pt}}

\setlength{\tabcolsep}{0pt}

\begin{table} 
    \caption[Cronograma de atividades]{Cronograma de Atividades. As marcações em preto indicam atividades que são priorizadas no período.}
    \begin{center}
        \begin{tabular}{|c|c|c|c|c|c|c|}
            \cline{2-7}
            \multicolumn{1}{l|}{} & \multicolumn{2}{c|}{2012} & \multicolumn{2}{c|}{2013} &        \multicolumn{2}{c|}{2014} \\
            \hline \ Atividade\ \ 
            & 1\textordmasculine\ S. & 2\textordmasculine\ S. 
            & 1\textordmasculine\ S. & 2\textordmasculine\ S. 
            & 1\textordmasculine\ S. & 2\textordmasculine\ S. \\
            \hline \hline                                        
            %     &       2012        &       2013         &       2014       \\
            1     &\y\y    &\y\y      &\x\x     &\x\x      &\x\x     &\x\x    \\ \hline
            2     &\x\y    &\x\x      &\x\x     &\x\x      &\x\x     &\x\x    \\ \hline
            3     &\x\x    &\y\y      &\y\y     &\r\r      &\r\r     &\x\x    \\ \hline
            4     &\x\x    &\y\y      &\y\y     &\r\r      &\r\r     &\x\x    \\ \hline
            5     &\x\x    &\x\x      &\y\y     &\x\x      &\x\x     &\x\x    \\ \hline
            6     &\x\x    &\x\x      &\y\y     &\x\x      &\x\x     &\x\x    \\ \hline
            7     &\x\x    &\x\x      &\x\y     &\r\x      &\x\x     &\x\x    \\ \hline
            8     &\x\x    &\x\x      &\x\x     &\r\r      &\x\x     &\x\x    \\ \hline
            9     &\x\x    &\x\x      &\x\x     &\r\r      &\x\x     &\x\x    \\ \hline
            10     &\x\x    &\x\x      &\x\x     &\x\r      &\r\r     &\x\x    \\ \hline
            11     &\x\x    &\x\y      &\y\y     &\r\r      &\r\r     &\x\x    \\ \hline
            12     &\x\x    &\x\x      &\x\x     &\x\r      &\r\r     &\x\x    \\ \hline
            %     &       2012        &       2013         &       2014       \\
        \end{tabular}
    \end{center}
    \label{t:atividades}
\end{table}
