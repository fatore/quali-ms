\chapter{Introdução}

\section{Contextualização e Motivação}

A nossa curiosidade em observar os eventos que nos cercam
aliada à nossa capacidade de raciocínio lógico nos
permitiram encontrar explicações para o ambiente em que
vivemos e para compreendermos melhor nós mesmos. Hoje, os
métodos de observação se encontram em um avançado estado de
desenvolvimento, sistemas computadorizados são capazes de
coletar um grande volume de dados em um piscar de
olhos. Mas atente para o termo dados. Dados por si somente
não conferem nenhum tipo de conhecimento. Tal riqueza só é
obtida quando aplicamos sobre as observações o nosso
raciocínio lógico.  Este recurso, por sua vez, não
acompanhou a evolução acentuada dos métodos de
observação, sendo que não há indícios que sejamos mais
mentalmente capazes do que gerações passadas. Logo, o antigo
método de aprendizado por meio da observação se depara com
um grande desafio, pois agora observamos muito mais
rapidamente do que explicamos.

Uma abordagem utilizada para amenizar este problema se
baseia em verificar se é realmente necessário analisar todos
os dados coletados para se obter novos conhecimentos. Como
muitas vezes os sistemas de observação não são capazes de
decernir se um dado é relevante ou não, então esta abordagem
normalmente reduz com sucesso o volume de dados a ser
investigado.  

Para ilustrar essa abordagem vamos imaginar a seguinte
situação: um geógrafo realizou um levantamento sobre um
conjunto de países onde notas de $0$ a $10$ foram atribuídas
em diferentes quesitos, de modo que quanto maior a nota
atribuída, melhor o país é naquele quesito\footnote{É
importante notar que este exemplo refere-se a uma
situação fictícia com propósito apenas ilustrativo.}. Com
base nessas notas ele deseja descobrir como esses países se
relacionam entre si. Deseja, por exemplo, definir grupos
de países similares e ser capaz de explicar porque esses
grupos foram formados. A Tabela~\ref{tab:at} apresenta o
resultado deste levantamento. Cada linha da tabela se
refere a uma observação, as colunas indicam características
dos países e a tabela como um todo é considerada um conjunto
de dados\footnote{No contexto deste documento, os termos
observação, instância, item e elemento podem ser
utilizados com o mesmo significado. O mesmo ocorre para
os termos característica, variável, dimensão e atributo.}.

Uma análise sobre a tabela não revela muito sobre as
relações entre os países. 

\begin{table}[htbp]
    \caption{Conjunto de dados fictício. Os dados foram
    gerados arbitrariamente e não apresentam necessariamente
    alguma relação com índices oficiais.}
    \begin{center}
        \begin{tabular}{|l|C{1.5cm}|C{1.5cm}|C{1.5cm}|C{1.5cm}|C{1.5cm}|C{1.5cm}|}
            \cline{2-7}
            \multicolumn{1}{c|}{\textbf{\textit{}}} & 
            \textbf{\textit{padrão}} & 
            \textbf{\textit{clima}} & 
            \textbf{\textit{gastronomia}} & 
            \textbf{\textit{segurança}} & 
            \textbf{\textit{recepção}} & 
            \textbf{\textit{infraestrutura}} \\ \hline
            \textbf{Itália} & 7 & 8 & 9 & 5 & 7 & 7 \\ \hline
            \textbf{Espanha} & 7 & 9 & 9 & 5 & 8 & 8 \\ \hline
            \textbf{Croácia} & 5 & 6 & 6 & 6 & 5 & 6 \\ \hline
            \textbf{Brasil} & 5 & 8 & 7 & 3 & 8 & 3 \\ \hline
            \textbf{Rússia} & 6 & 2 & 2 & 3 & 3 & 6 \\ \hline
            \textbf{Alemanha} & 8 & 3 & 2 & 8 & 3 & 9 \\ \hline
            \textbf{Turquia} & 5 & 8 & 9 & 3 & 9 & 3 \\ \hline
            \textbf{Marrocos} & 4 & 7 & 8 & 2 & 9 & 2 \\ \hline
            \textbf{Peru} & 5 & 6 & 6 & 3 & 6 & 4 \\ \hline
            \textbf{Nigéria} & 2 & 4 & 4 & 2 & 7 & 2 \\ \hline
            \textbf{França} & 8 & 4 & 7 & 7 & 1 & 8 \\ \hline
            \textbf{México} & 2 & 5 & 5 & 2 & 7 & 3 \\ \hline
        \end{tabular}
    \end{center}
    \label{tab:at}
\end{table}

% Falar de redução é de boa.
% Mas como eu introduzo a tal modificação.
% Como eu junto as duas coisas?
% Eu sei dar exemplos: arrastar um país que ele sabe que é
% parecido para perto de outro.
% Mas como isso melhora o dataset? 


Do mesmo modo que métodos computacionais são utilizados para
desenvolver sistemas de observação que coletam dados
automaticamente, métodos computacionais têm sido utilizados para
ensinar sistemas a pensar como o ser humano e serem capazes
de extrair conhecimento dos dados de forma automatizada.
Devido ao fato desses sistemas exigirem um certo tipo de
treinamento, eles são ditos métodos de aprendizado de
máquina. No entanto, mesmo esses métodos tem
dificuldade em processar grandes volumes de dados, fazendo
com que o resultado obtido seja impreciso.

Assim, uma forma de avaliar as contribuições deste trabalho
é justamente comparar o resultado obtido pelos métodos de
aprendizado de máquina antes e após a transformação
interativa do espaço de atributos. Para se realizar uma
avaliação justa, outros métodos de transformação do espaço,
não necessariamente interativa, podem ser comparados.

Este capítulo teve como objetivo apresentar uma abstração do
trabalho de mestrado aqui proposto e introduzir o conceito
de transformação do espaço de atributos. No próximo capítulo
apresenta-se um levantamento sobre outros trabalhos que
realizam este tipo de transformação, seja ela interativa ou
não. No terceiro e último capítulo discute-se com mais
detalhes a proposta deste trabalho, a metodologia a ser
adotada e um cronograma de atividades.
