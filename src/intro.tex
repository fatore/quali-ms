\chapter{Introdução}

\section{Contextualização e Motivação}

% Introduzo conjuntos de dados de alta dimensionalidade. (Imagem, Texto, Microarray)
% Grandes conjuntos de dados com centenas de dimensões são comuns.

% Cito algumas aplicações:

% OCR and handwriting recognition, industrial visual inspection and identifkation, surveillance and identification, access control using biometric systems and chip card technology, head-finder, eye-tracker, passenger counting system, aircraft docking systems as well as driver assistance tasks in the automotive domain.

% Introduzo o problema de trabalhar com esses dados.

% Menciono as técnicas que lidam com esses dados. (Agrupadores, Classificadores, Visualizações, Data Mining em geral)

% Explico a MD (the lack of data separation in a high dimensional space) e como ela afeta o desempenho das técnicas que operam sobre os dados.
% #examples needed to train classifier function grows exponentially with dimensions

% ? Técnicas de visualização multidimensional são ferramentas importantes ao se abordar este problema. Elas permitem uma interação direta com o usuário e provém feedback imediato. No entanto, as técnicas de visualização tradicionais, como Scatter Plot Matrices, PC e glifos não escalam bem. Posso até exemplificar com números, com 200 teria 200 eixos em PC, 40,000 gráficos em SM, etc. Poucas afirmam ser capazes de lidar com centenas de dimensões. 

% Falo sobre a necessidade de se evitar dados redundantes e irrelevantes.

% Falo de métodos de redução automática.

% Aponto o problema dos métodos automáticos.

% Apresento métodos interativos e visuais como uma alternativa para a exploração dos dados e também para a compreensão dos resultados das ferramentas automáticas. Pois representações gráficas permitem que o usuário aplique suas habilidades perceptivas e o conhecimento sobre o domínio.

% Destaco as projeções.

% Digo o que vou fazer. "vou fazer redução de dim visual"

\section{Organização da Monografia}

O restante desta monografia está estruturado da seguinte maneira:

\begin{itemize}
        
  \item No Capítulo~\ref{chap:revisao} apresenta-se um levantamento sobre os trabalhos que buscam de algum modo fazer uso de representações visuais para a execução da tarefa de redução de dimensionalidade. 

  \item No Capítulo~\ref{chap:proposta} discute-se com mais detalhes a proposta de trabalho e a metodologia adotada. Apresenta-se também um cronograma das atividades necessárias para a conclusão do trabalho.

\end{itemize}

