\chapter{Introdução}

% Exploração dos dados
A exploração de conjuntos de dados é um problema abordado
com frequência em computação, tanto na área acadêmica quanto
na indústria~\cite{Ngai2009,Harding2006}. Tal exploração tem
como objetivo uma melhor compreensão dos fenômenos que
afetam a sociedade nos mais variados aspectos. Com base nos
conhecimentos adquiridos durante a exploração é possível 
melhorar o processo de tomadas de decisões, como
previsão de condições climáticas, diagnósticos de doenças,
detecções de fraude, análise de mercado, etc.

% Dados multidimensionais 
As investigações sobre os dados são precedidas pelas etapas
de coleta e armazenamento dos dados, as quais podem ser
realizadas por sensores, sistemas de monitoramento,
simulações computacionais ou aplicações diversas que
utilizam banco de dados. Nessas etapas, busca-se registrar o
máximo de detalhes sobre o fenômeno observado. No caso de
análises climáticas, por exemplo, poderiam ser observadas
variáveis como temperatura, velocidade do vento, umidade do
ar, etc. As observações poderiam ser realizadas em
diferentes posições geográficas, onde para cada posição
teria-se uma coleção de variáveis. Comumente, denomina-se
cada uma dessas coleções como uma instância de dados.
Assim, sabendo que cada variável observada também pode ser
chamada de dimensão, quando diversas instâncias de dados são
coletadas obtém-se um conjunto de dados multidimensional. 

% Problema de dados multidimensionais
Explorar conjuntos de dados multidimensionais envolve uma
série de desafios. Um deles é a dificuldade em se
identificar quais dos dados coletados são realmente
relevantes para as análises. Isso faz com que, muitas vezes,
se utilize todas as variáveis observadas nas investigações,
sejam elas relevantes ou não. Nessas condições, eleva-se a
complexidade das análises e também o custo computacional dos
métodos que são aplicados sobre os dados~\cite{Beyer1999}.
Um outro problema é a falta de garantias de que os fatores
fundamentais para a compreensão do problema tenham sido
coletados. A maioria das aplicações está sujeita a essa
situação, pois os sistemas de coleta de dados são
suscetíveis a falhas e não reconhecem facilmente fatores
subjetivos. 

% Solução para o problema: transformar os dados Primeira
Um modo de contornar esses problemas é transformar os dados
para torná-los mais representativos para a execução das
tarefas seguintes. A transformação mais comumente aplicada
sobre os dados é a redução de dimensionalidade. Esta tem
como objetivo encontrar o menor espaço dimensional que é
capaz de descrever os dados mantendo informações que são
relevantes segundo algum critério. O processo de redução
pode ser realizado tanto pela eliminação de dimensões
irrelevantes ou redundantes quanto pela combinação entre
dimensões. 

% Problema dos métodos clássicos de redução 
Um dos problemas dos métodos de redução de dimensionalidade
é que o conceito de relevância é subjetivo e pode variar de
acordo com a aplicação. Além disso, os métodos tradicionais
de redução apresentam uma natureza dita ``caixa-preta'',
pois o usuário inspeciona apenas os dados de entrada e
saída, desconhecendo o processamento realizado internamente.
Esta natureza torna esses métodos pouco compreensivos e,
além disso, proíbe que o usuário contribua com a sua
experiência na área. Buscando tratar essa limitação novos
métodos têm sido propostos. Eles permitem que o usuário
guie o processo de redução por meio da interação com
representações gráficas dos dados.

% Vantagens métodos visuais
Técnicas baseadas em visualizações que permitem a
interação do usuário têm sido aplicadas não somente em
tarefas de redução de dimensionalidade, mas em diversas
áreas de exploração de dados. Elas ganharam grande
popularidade nos últimos anos~\cite{State2012} e propiciaram
a firmação da área de visualização de
informação~\cite{Keim2002}. Grande parte do sucesso desta
recente área pode ser atribuído ao uso efetivo da capacidade
preemptiva da visão humana na exploração dos dados. Foi
demonstrado que quando os dados são representados por 
gráficos, o ser humano é capaz de detectar e reconhecer
padrões de forma mais fácil e rápida~\cite{Healey1995},
mesmo em grandes conjuntos de dados~\cite{Fodor2002}. 

% Métodos visuais viabilizam a segunda abordagem
No entanto, utilizar a capacidade preemptiva da visão humana
não é a única vantagem das técnicas de visualização de
informação. Ao permitirem que o usuário participe ativamente
na geração dos resultados, essas técnicas viabilizam novas
abordagens para se explorar e transformar os dados. A
construção interativa de dimensões é uma dessas novas
abordagens e refere-se justamente a uma alternativa à
redução de dimensionalidade para se transformar os conjuntos
de dados.

% Motivação construção 
A necessidade de se construir novas dimensões surge em
situações em que a redução de dimensionalidade, seja ela
automática ou interativa, não é suficiente para tornar os
dados mais representativos.  Isso ocorre quando fatores
fundamentais para a compreensão do problema não foram
capturados na etapa de coleta dos dados. Em situações como
essa, o problema só poderá ser descrito por completo ao se
agregar o conhecimento do usuário nos dados.

% Limitações trabalhos relacionados - gancho para objetivo
A pesquisa em construção interativa de dimensões é muito
recente, sendo que a principal contribuição data neste mesmo
ano~\cite{Gladys2013}. As técnicas propostas ainda
apresentam diversas limitações, o que abre espaço para
trabalhos futuros. As técnicas de redução de
dimensionalidade têm sido discutidas há mais de um
século~\cite{Pearson1901} e apresentam as limitações
discutidas anteriormente. Os métodos de redução interativa
não são tão recentes quanto os de construção de dimensões.
Porém, apresentam limitações tanto em relação às
visualizações em que se baseiam quanto aos mecanismos de
interação propostos.

% Objetivo do trabalho
Deste modo, este trabalho de mestrado tem como objetivo
desenvolver uma técnica para transformação de dados
multidimensionais que atenda às duas situações descritas
acima.  A primeira, em que é necessário reduzir a
dimensionalidade dos dados para se eliminar variáveis
irrelevantes ou redundantes. E a segunda, em que novas
dimensões devem ser construídas com base no conhecimento do
usuário para representar informações ausentes nos dados. A
abordagem proposta leva em consideração as limitações dos
trabalhos do atual estado da arte e se baseia na
simplicidade das visualizações e dos mecanismos de interação
para superá-las. 

% Resultados esperados e forma de avaliação 
A obtenção de conjuntos de dados representativos, isto é,
conjuntos que não contém atributos irrelevantes ou
redundantes e que refletem a opinião do usuário,
proporcionará uma análise mais efetiva dos dados. Será
possível, por exemplo, melhorar o desempenho de agrupadores
e classificadores de dados. Pretende-se avaliar as
contribuições deste trabalho justamente pela quantificação
do desempenho de tais métodos ao precedê-los pela técnica
desenvolvida, seguida de uma comparação com técnicas já
estabelecidas na literatura.

% Organização monografia
O restante deste documento está organizado da seguinte
maneira. No Capítulo~\ref{chap:revisao}, apresenta-se um
levantamento sobre outros trabalhos que também buscam
transformar conjuntos de dados para torná-los mais
representativos para o problema em estudo. Por fim, no
Capítulo~\ref{chap:proposta}, discute-se com mais detalhes a
proposta deste trabalho e a metodologia a ser adotada. 

