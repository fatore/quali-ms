\begin{resumo}
  %
  Em aplicações de análise de dados, raramente se conhece a dimensionalidade intrínseca dos dados, isto é, o conjunto de variáveis observadas que são realmente relevantes para a compreensão do fenômeno estudado.
  %
  Métodos automáticos são frequentemente utilizados para compor subconjuntos de atributos, ou combinações entre eles, que agreguem a maior parte da informação contida nos dados. 
  %
  Entretanto, esses métodos evitam ao máximo a interação do usuário, o que além de tornar o processo pouco intuitivo, impede que o usuário modifique os resultados de acordo com a sua experiência na área. 
  %
  Técnicas de visualização computacional têm sido utilizadas com sucesso na análise exploratória de conjuntos de dados, pois permitem que o usuário utilize sua percepção visual para detectar padrões e seu conhecimento sobre o domínio para interagir com os dados e orientar as análises. 
  %
  Este projeto de mestrado propõe o uso de projeções multidimensionais coordenadas entre itens e dimensões para a execução da tarefa de redução de dimensionalidade de forma mais intuitiva, ágil e confiável. Além disso, propõe-se um método que permite ao usuário modificar os resultados obtidos, quando necessário, por meio da transformação do espaço de atributos.
  %
\end{resumo}
