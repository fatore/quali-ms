\begin{resumo}
%
A exploração de conjuntos de dados é um problema abordado
com frequência em computação e tem como objetivo uma melhor
compreensão de fenômenos simulados ou medidos.
%
Tal atividade é precedida pelas etapas de coleta e
armazenamento de dados que buscam registrar o máximo
de detalhes sobre o fenômeno observado.
%
A exploração efetiva dos dados envolve uma série de
problemas. 
%
Um deles é a dificuldade em identificar quais
dados são realmente relevantes para as
análises. 
%
Outro problema está relacionado com a falta de
garantias de que os fatores fundamentais para a compreensão
do problema tenham sido coletados.
% 
A transformação interativa de dados é uma abordagem que
utiliza técnicas de visualização computacional para 
resolver ou minimizar esses problemas. 
%
No entanto, os trabalhos disponíveis na literatura
possuem limitações, como interfaces demasiadamente
complexas e mecanismos de interação pouco flexíveis. 
%
Assim, este projeto tem como objetivo desenvolver novas
técnicas visuais interativas para a transformação de dados
multidimensionais.
%
A metodologia proposta se baseia no uso de \emph{biplots} e na
ação conjunta dos mecanismos de interação para superar as
limitações das técnicas do estado da arte.
%
Acredita-se que ao utilizar as abordagens propostas 
será possível tornar os conjuntos de dados mais
representativos. 
%
Logo, atividades de exploração poderão ser
realizadas com mais eficácia e retornarão melhores
resultados.
\end{resumo}
